% Appendix A

\chapter{\texttt{R}-Functions} % Main appendix title

\label{AppendixA} % For referencing this appendix elsewhere, use \ref{AppendixA}

\lhead{Appendix A. \emph{\texttt{R}-Functions}} % This is for the header on each page - perhaps a shortened title

In order to make some of the \texttt{R}-codes readable, most of the repeated operations have been wrapped up into functions that are used throughout the dissertation.
\section{Matrices of Basis Functions and Model Selection}
  \\
  \\[1ex]
  \mbox{
    \begin{tabular}{ c c c c}
    \hline \\[-0.75ex]
    \hspace{1cm} \texttt{Gaussian\_bsplines} &  \hspace{.5cm} \textit{Gaussian Basis} functions with \textit{B-Splines} & & \\
    \\[-0.75ex] 
    \hline\\[-5ex]
    \end{tabular}%
  }
\subsection*{Description}
This function is used to compute a matrix of \textit{Gaussian Basis} functions with \textit{B-Splines}. Its arguments are:
\begin{itemize}
\item \texttt{tt} being the vector of values $\left\lbrace t_{1},\dots,t_{J} \right\rbrace \in \mathcal{T}$;
\item \texttt{m} represents the number of basis functions applied to the function.
\end{itemize}

\subsection*{\texttt{R}-Code}
\begin{lstlisting}
  Gaussian_bsplines = function(tt,m){
    
    range <- diff(range(tt))
    kn <- seq(min(tt) - (range/(m-3))*3, max(tt) + (range/(m-3))*3, by = range/(m-3))
    myu <- kn[3:(m+2)]
    h <- diff(kn,lag = 2)/3
    
    B <- matrix(0,length(tt),(m))
    for (j in 1:m){
      B[,j] <- exp(-0.5*(tt-myu[j])^2/(h[1]^2))
    }
    return(B)
  }
 \end{lstlisting}
\vspace{0.5cm}
  \mbox{
    \begin{tabular}{ c c c c}
    \hline \\[-0.75ex]
    \hspace{.5cm} \texttt{Gaussian\_kmeans} &  \hspace{1cm} \textit{Gaussian Basis} functions with \texttt{K-Means} & & \\
    \\[-0.75ex] 
    \hline\\[-1ex]
    \end{tabular}%
  }
  \subsection*{Description}
  This function is used to compute a matrix of \textit{Gaussian Basis} functions using \textit{K-means}. Its arugments are:
 \begin{itemize}
 \item \texttt{tt} is used to specify the vector of values $\left\lbrace t_{1},\dots,t_{J} \right\rbrace \in \mathcal{T}$;
 \item \texttt{m} is used to specify the number of basis functions applied to the function;
 \item \texttt{nyu} is used to specify the hyperparameter.
 \end{itemize} The clustering method used is the one developed by \cite{Hartigan1979}. 
 \subsection*{\texttt{R}-Code}
 \begin{lstlisting}
  Gaussian_kmeans = function(tt,m,nyu){
     
    k <- kmeans(tt, centers = m,algorithm = "Hartigan-Wong")
    myu <- as.vector(k$centers)
    h <- k$withinss/k$size    
     
    B <- matrix(0,length(tt),(m))
    for (j in 1:m){
      B[,j] <- exp(-0.5*(tt-myu[j])^2/(h[j]*nyu))
    }
    return(B)
  }
  \end{lstlisting}
\clearpage
\\
 \\
 \\[3ex]
 \mbox{
   \begin{tabular}{ c c c c}
   \hline \\[-0.75ex]
   \hspace{1cm} \texttt{Bsplines\_FDA} &  \hspace{1cm} \textit{B-Splines Basis} functions & \hspace{1cm} & \hspace{1.5cm} \\
   \\[-0.75ex] 
   \hline\\[-4ex]
   \end{tabular}%
 }
\subsection*{Description}
This function is used to generate a matrix of \textit{B-Splines Basis} functions. It uses the \texttt{R}-package \texttt{fda}. Its arguments are:
\begin{itemize}
\item \texttt{tt} is used to specify the vector of values $\left\lbrace t_{1},\dots,t_{J} \right\rbrace \in \mathcal{T}$;
\item \texttt{m} is used to specify the number of basis functions applied to the function.
\item \texttt{norder} is used to specify the order of the \textit{B-Splines}
\end{itemize}
\subsection*{\texttt{R}-Code}
\begin{lstlisting}
Bsplines_FDA <- function(tt,m,norder=4){
  require(fda)
  basis = create.bspline.basis(rangeval = range(tt),nbasis = m,norder)
  B <- eval.basis(evalarg = tt,basisobj = basis)
  return(B)
} 
\end{lstlisting} 
\vspace{0.5cm}
 \mbox{
   \begin{tabular}{ c c c c}
   \hline \\[-0.75ex]
   \hspace{1cm} \texttt{Fourier\_FDA} &  \hspace{1cm} \textit{Fourier Basis} functions & \hspace{2cm} & \hspace{1.5cm} \\
   \\[-0.75ex] 
   \hline\\[-2ex]
   \end{tabular}%
 }
\subsection*{Description}
This function is used to generate a matrix of \textit{Fourier Basis} functions. It uses the \texttt{R}-package \texttt{fda}. Its arguments are:
\begin{itemize}
\item \texttt{tt} is used to specify the vector of values $\left\lbrace t_{1},\dots,t_{J} \right\rbrace \in \mathcal{T}$;
\item \texttt{m} is used to specify the number of basis functions applied to the function.
\end{itemize}
\clearpage
\subsection*{\texttt{R}-Code}
\begin{lstlisting}
Fourier_FDA <- function(tt,m){
  require(fda)
  if((m %% 2)==0) {m <- m + 1} else {m <- m}
  basis = create.fourier.basis(rangeval = range(tt),nbasis = m)
  B <- eval.basis(evalarg = tt,basisobj = basis)
  return(B)
}
\end{lstlisting}
\vspace{0.5cm}
\mbox{
   \begin{tabular}{ c c c c}
   \hline \\[-0.75ex]
   \hspace{.5cm} \texttt{\texttt{Pen\_Max\_Likelihood}} &  \hspace{1cm} \textit{Penalized Maximum Likelihood} estimate &\hspace{-0.5cm} & \\
   \\[-0.75ex] 
   \hline\\[-2ex]
   \end{tabular}%
}
\subsection*{Description}
This function is used to compute the \textit{Penalized Maximum Likelihood} estimate. Its arguments are:
\begin{itemize}
\item \texttt{B} is used to specify the matrix of basis functions;
\item \texttt{n} is used to specify the number of basis functions;
\item \texttt{lambda} is used to specify log${}_{10}(\lambda)$;
\item \texttt{y} is used for the vector of observed values.
\end{itemize}
\subsection*{\texttt{R}-Code}
\begin{lstlisting}
Pen_Max_Likelihood <- function(B, n, lambda, y){
  D <- matrix(0,(n-2),n)
  D[1, ] <- c(1,-2,1,rep(0,(n-3)))
  for (i in 1:(n-4)) {
    D[(i+1), ] <- c(rep(0,i),1,-2,1,rep(0,(n-3)-i))
  }
  D[(n-2), ] <- c(rep(0,(n-3)),1,-2,1)
  K <- t(D)%*%D
  
  lamda <- 10^(lambda)
  sigma <- 2
  sigma1 <- 1
  
  while((sigma-sigma1)^2 > 1e-7){
    Binv <- solve(t(B)%*%B+ncol(train.temp)*(lamda)*(sigma)*K,diag(ncol(K)))
    w <- (Binv)%*%t(B)%*%y[1,]
    sigma1 <- sigma
    sigma1 <- as.vector(sigma1)
    sigma <- (1/ncol(train.temp))*t(y[1,]-B%*%w)%*%(y[1,]-B%*%w)
    sigma <- as.vector(sigma)       
  }
  list(lamda=lamda,sigma=sigma,K=K,w=w)
}
\end{lstlisting}

\section{Model Criterion}
\vspace{0.5cm}
 \mbox{
   \begin{tabular}{ c c c c}
   \hline \\[-0.75ex]
   \hspace{1cm} \texttt{gcv\_fun} &  \hspace{1cm} \textit{Generalized Cross-Validation} criterion& \hspace{2cm} & \hspace{1cm}\\
   \\[-0.75ex] 
   \hline\\[-1ex]
   \end{tabular}%
 }
\subsection*{Description}
This function is used to compute the \textit{Generalized Cross-Validation} criterion for model evaluation. Its arguments are:
\begin{itemize}
\item \texttt{ob} is used to specify the object created from the \texttt{Pen\_Max\_Likelihood} function;
\item \texttt{y} is used for the vector of observed values;
\item \texttt{tt} is used to specify the vector of values $\left\lbrace t_{1},\dots,t_{J} \right\rbrace \in \mathcal{T}$.
\end{itemize}

\subsection*{\texttt{R}-Code}
\begin{lstlisting}
gcv_fun <- function(tt, y, ob){
  Binv <- solve(t(B)%*%B+length(y)*(ob$lamda)*(ob$sigma)*ob$K,diag(n))
  H <- B%*%(Binv)%*%t(B)
  yhat <- H%*%y[1,]
  den = 1 - sum(diag(H))/length(tt) # load(matrixcalc)
  y.diff = yhat - y[1,]
  return(mean((y.diff/den)^2))
}
\end{lstlisting}
\vspace{0.5cm}
 \mbox{
   \begin{tabular}{ c c c c}
   \hline \\[-0.75ex]
   \hspace{1cm} \texttt{gic\_fun} &  \hspace{1cm} \textit{Generalized Information Criterion} & \hspace{2.5cm} & \\
   \\[-0.75ex] 
   \hline\\[-1ex]
   \end{tabular}%
 }
\subsection*{Description} 
This function is used to compute the \textit{Generalized Information Criterion} for model evaluation. Its arguments are:
\begin{itemize}
\item \texttt{ob} is used to specify the object created from the \texttt{Pen\_Max\_Likelihood} function;
\item \texttt{y} is used for the vector of observed values;
\item \texttt{tt} is used to specify the vector of values $\left\lbrace t_{1},\dots,t_{J} \right\rbrace \in \mathcal{T}$.
\end{itemize}
\subsection*{\texttt{R}-Code}
\begin{lstlisting}
gic_fun <- function(y,ob,n){
  gamma <- diag(as.vector(y[1,]-B%*%ob$w))
  one <- rep(1,length(y))
  
  R1 <- rbind(t(B)%*%B+length(y)*(ob$lamda)*(ob$sigma)*ob$K,t(one)%*%gamma%*%B/(ob$sigma))
  R2 <- rbind(t(B)%*%gamma%*%one/(ob$sigma),length(y)/(2*(ob$sigma)))
  R <- cbind(R1,R2)
  R <- R/(length(y)*(ob$sigma))
  if(det(R) < 10^(103)) {Rinv <- solve(R,diag(n+1))} else {Rinv <- NA}
  
  Q1 <- rbind(t(B)%*%(gamma)^2%*%B/(ob$sigma)-(ob$lamda)*ob$K%*%ob$w%*%t(one)%*%gamma%*%B,t(one)%*%(gamma)^3%*%B/(2*(ob$sigma)^2)-t(one)%*%gamma%*%B/(2*(ob$sigma)))
  Q2 <- rbind(t(B)%*%(gamma)^3%*%one/(2*(ob$sigma)^2)-t(B)%*%gamma%*%one/(2*(ob$sigma)),t(one)%*%(gamma)^4%*%one/(4*(ob$sigma)^3)-length(y)/(4*(ob$sigma)))
  Q <- cbind(Q1,Q2)
  Q <- Q/(length(y)*(ob$sigma))
  
  V <- ifelse(det(R) < 10^(103) & all(!is.na(Rinv)), length(y)*(log(2*pi)+1)+length(y)*log(ob$sigma)+2*sum(diag(Rinv%*%Q)), NA)
  return(V)
}
\end{lstlisting}
\vspace{0.5cm}
 \mbox{
   \begin{tabular}{ c c c c}
   \hline \\[-0.75ex]
   \hspace{1cm} \texttt{mAIC\_fun} &  \hspace{1cm} \textit{modified Akaike Information Criterion} & \hspace{1.5cm} & \\
   \\[-0.75ex] 
   \hline\\[-1ex]
   \end{tabular}%
 }
\subsection*{Description} 
This function is used to compute the \textit{modified AIC} method for model evaluation. Its arguments are:
\begin{itemize}
\item \texttt{ob} is used to specify the object created from the \texttt{Pen\_Max\_Likelihood} function;
\item \texttt{y} is used for the vector of observed values;
\item \texttt{tt} is used to specify the vector of values $\left\lbrace t_{1},\dots,t_{J} \right\rbrace \in \mathcal{T}$.
\end{itemize}
\subsection*{\texttt{R}-Code}
\begin{lstlisting}
mAIC_fun <- function(ob,y,n){
  Binv <- solve(t(B)%*%B+length(y)*(ob$lamda)*(ob$sigma)*ob$K,diag(n))
  H <- B%*%(Binv)%*%t(B)
  return(length(y)*(log(2*pi)+1)+length(y)*log(ob$sigma)+2*sum(diag(H)))
}
\end{lstlisting}
\vspace{0.5cm}
 \mbox{
   \begin{tabular}{ c c c c}
   \hline \\[-0.75ex]
   \hspace{.5cm} \texttt{gbic\_fun} &  \hspace{1cm} \textit{Generalized Bayesian Information Criterion} &\hspace{1cm} & \\
   \\[-0.75ex] 
   \hline\\[-1ex]
   \end{tabular}%
 }
\subsection*{Description}
This function is used to compute the \textit{Generalized Bayesian Information Criterion} for model evaluation. Its arguments are:
\begin{itemize}
\item \texttt{ob} is used to specify the object created from the \texttt{Pen\_Max\_Likelihood} function;
\item \texttt{y} is used for the vector of observed values;
\item \texttt{tt} is used to specify the vector of values $\left\lbrace t_{1},\dots,t_{J} \right\rbrace \in \mathcal{T}$.
\end{itemize}
The \texttt{R}-code is as follows:
\begin{lstlisting}
gbic <- function(y,ob,n){
  gamma <- as.vector(y[1,]-B%*%ob$w)
  
  Q1 <- rbind(t(B)%*%B+length(y)*(ob$lamda)*(ob$sigma)*ob$K,t(gamma)%*%B/(ob$sigma))
  Q2 <- rbind(t(B)%*%gamma/(ob$sigma),length(y)/(2*(ob$sigma)))
  Q <- cbind(Q1,Q2)
  Q <- Q/(length(y)*(ob$sigma))
  Q.det <- det(Q)
  vec <- eigen(ob$K)$values
  vec <- vec[vec >= 0]
  
  return((length(y)+n-1)*log(ob$sigma) + length(y)*(ob$lamda)*(ob$sigma)*t(ob$w)%*%ob$K%*%ob$w/(ob$sigma) + length(y) + (length(y)-3)*log(2*pi)+
    3*log(length(y)) + log(Q.det) - log(prod(vec)) - (n-1)*log((ob$lamda)*(ob$sigma)))
}
\end{lstlisting}
\clearpage
 \mbox{
   \begin{tabular}{ c c c c}
   \hline \\[-0.75ex]
   \hspace{1cm} \texttt{GCV.Gauss\_bs} &  \hspace{1cm} \textit{Generalized Cross-Validation}$^2$ &\hspace{2cm} & \\
   \\[-0.75ex] 
   \hline\\[-4ex]
   \end{tabular}%
 }
\subsection*{Description}
  This function computes the \textit{Generalized Cross-Validation} criterion using the \textit{Least Squares} method without a smoothing parameter based Gaussian basis function with \textit{B-Splines}. Its arguments are:
  \begin{itemize}
  \item \texttt{dat} is used to specify the $N \times J$ matrix of observations;
  \item \texttt{tt} is used to specify the vector of values $\left\lbrace t_{1},\dots,t_{J} \right\rbrace \in \mathcal{T}$;
  \item \texttt{m} is used to specify the number of basis functions applied to the function.
  \end{itemize} 
  This function is used run for illustrative purpose (see section~\ref{foreach})
  \subsection*{\texttt{R}-Code}
  \begin{lstlisting}
   S = NULL
   GCV.Gauss_bs = function(dat,tt,m){
     
     range <- diff(range(tt))
     kn <- seq(min(tt) - (range/(m-3))*3, max(tt) + (range/(m-3))*3, by = range/(m-3))
     myu <- kn[3:(m+2)]
     h <- diff(kn,lag = 2)/3
      
     B <- matrix(0,length(tt),(m))
     for (j in 1:m){
       B[,j] <- exp(-0.5*(tt-myu[j])^2/(h[1]^2))
     }
     Binv <- solve(t(B)%*%B,diag(m))
     S <- B%*%Binv%*%t(B)
     xhat <- S%*%dat
     den <- 1 - sum(diag(S))/length(tt)
     x.diff <- xhat - dat
     return(mean((x.diff/den)^2)) # GCV value
  }
    \end{lstlisting}
